 \documentclass[journal,12pt,twocolumn]{IEEEtran}

\usepackage{setspace}
\usepackage{gensymb}
\singlespacing
\usepackage[cmex10]{amsmath}

\usepackage{amsthm}

\usepackage{mathrsfs}
\usepackage{txfonts}
\usepackage{stfloats}
\usepackage{bm}
\usepackage{cite}
\usepackage{cases}
\usepackage{subfig}

\usepackage{longtable}
\usepackage{multirow}

\usepackage{enumitem}
\usepackage{mathtools}
\usepackage{steinmetz}
\usepackage{tikz}
\usepackage{circuitikz}
\usepackage{verbatim}
\usepackage{tfrupee}
\usepackage[breaklinks=true]{hyperref}
\usepackage{graphicx}
\usepackage{tkz-euclide}

\usetikzlibrary{calc,math}
\usepackage{listings}
    \usepackage{color}                                            %%
    \usepackage{array}                                            %%
    \usepackage{longtable}                                        %%
    \usepackage{calc}                                             %%
    \usepackage{multirow}                                         %%
    \usepackage{hhline}                                           %%
    \usepackage{ifthen}                                           %%
    \usepackage{lscape}     
\usepackage{multicol}
\usepackage{chngcntr}

\DeclareMathOperator*{\Res}{Res}

\renewcommand\thesection{\arabic{section}}
\renewcommand\thesubsection{\thesection.\arabic{subsection}}
\renewcommand\thesubsubsection{\thesubsection.\arabic{subsubsection}}

\renewcommand\thesectiondis{\arabic{section}}
\renewcommand\thesubsectiondis{\thesectiondis.\arabic{subsection}}
\renewcommand\thesubsubsectiondis{\thesubsectiondis.\arabic{subsubsection}}


\hyphenation{op-tical net-works semi-conduc-tor}
\def\inputGnumericTable{}                                 %%

\lstset{
%language=C,
frame=single, 
breaklines=true,
columns=fullflexible
}
\begin{document}


\newtheorem{theorem}{Theorem}[section]
\newtheorem{problem}{Problem}
\newtheorem{proposition}{Proposition}[section]
\newtheorem{lemma}{Lemma}[section]
\newtheorem{corollary}[theorem]{Corollary}
\newtheorem{example}{Example}[section]
\newtheorem{definition}[problem]{Definition}

\newcommand{\BEQA}{\begin{eqnarray}}
\newcommand{\EEQA}{\end{eqnarray}}
\newcommand{\define}{\stackrel{\triangle}{=}}
\bibliographystyle{IEEEtran}
\raggedbottom
\setlength{\parindent}{0pt}
\providecommand{\mbf}{\mathbf}
\providecommand{\pr}[1]{\ensuremath{\Pr\left(#1\right)}}
\providecommand{\qfunc}[1]{\ensuremath{Q\left(#1\right)}}
\providecommand{\sbrak}[1]{\ensuremath{{}\left[#1\right]}}
\providecommand{\lsbrak}[1]{\ensuremath{{}\left[#1\right.}}
\providecommand{\rsbrak}[1]{\ensuremath{{}\left.#1\right]}}
\providecommand{\brak}[1]{\ensuremath{\left(#1\right)}}
\providecommand{\lbrak}[1]{\ensuremath{\left(#1\right.}}
\providecommand{\rbrak}[1]{\ensuremath{\left.#1\right)}}
\providecommand{\cbrak}[1]{\ensuremath{\left\{#1\right\}}}
\providecommand{\lcbrak}[1]{\ensuremath{\left\{#1\right.}}
\providecommand{\rcbrak}[1]{\ensuremath{\left.#1\right\}}}
\theoremstyle{remark}
\newtheorem{rem}{Remark}
\newcommand{\sgn}{\mathop{\mathrm{sgn}}}
\providecommand{\abs}[1]{\(\left\vert#1\right\vert\)}
\providecommand{\res}[1]{\Res\displaylimits_{#1}} 
\providecommand{\norm}[1]{\(\left\lVert#1\right\rVert\)}
%\providecommand{\norm}[1]{\lVert#1\rVert}
\providecommand{\mtx}[1]{\mathbf{#1}}
\providecommand{\mean}[1]{E\(\left[ #1 \right]\)}
\providecommand{\fourier}{\overset{\mathcal{F}}{ \rightleftharpoons}}
%\providecommand{\hilbert}{\overset{\mathcal{H}}{ \rightleftharpoons}}
\providecommand{\system}{\overset{\mathcal{H}}{ \longleftrightarrow}}
	%\newcommand{\solution}[2]{\textbf{Solution:}{#1}}
\newcommand{\solution}{\noindent \textbf{Solution: }}
\newcommand{\cosec}{\,\text{cosec}\,}
\providecommand{\dec}[2]{\ensuremath{\overset{#1}{\underset{#2}{\gtrless}}}}
\newcommand{\myvec}[1]{\ensuremath{\begin{pmatrix}#1\end{pmatrix}}}
\newcommand{\mydet}[1]{\ensuremath{}}
\numberwithin{equation}{subsection}
\makeatletter
\@addtoreset{figure}{problem}
\makeatother
\let\StandardTheFigure\thefigure
\let\vec\mathbf
\renewcommand{\thefigure}{\theproblem}
\def\putbox#1#2#3{\makebox[0in][l]{\makebox[#1][l]{}\raisebox{\baselineskip}[0in][0in]{\raisebox{#2}[0in][0in]{#3}}}}
     \def\rightbox#1{\makebox[0in][r]{#1}}
     \def\centbox#1{\makebox[0in]{#1}}
     \def\topbox#1{\raisebox{-\baselineskip}[0in][0in]{#1}}
     \def\midbox#1{\raisebox{-0.5\baselineskip}[0in][0in]{#1}}
\vspace{3cm}
\title{AI1103 - Assignment 3}
\author{Vishwanath Hurakadli - AI20BTECH11023}
\maketitle
\newpage
\bigskip
\renewcommand{\thefigure}{\theenumi}
\renewcommand{\thetable}{\theenumi}
%
Download latex codes from 
%
\begin{lstlisting}
https://github.com/vishwahurakadli/AI1103/blob/main/Assignment_3/Assignment_3.tex
\end{lstlisting}
\begin{enumerate}
\item Gate MA(2015) Q26:\\
Let X and Y be two random variables having the joint probability density function\\
$f_{XY}(x,y)= \begin{cases}
2, \text{ if } 0<x<y<1\\
0,\text{ otherwise }
\end{cases}$\\
Then the conditional probability \(\Pr(X\leqslant2/3|Y\leqslant3/4)\) is equal to
\begin{enumerate}[]
\begin{multicols}{4}
\setlength\itemsep{0.1em}
\item \(\frac{5}{9}\) \label{option A}
\item \(\frac{2}{3}\) \label{option B}
\item \(\frac{7}{9}\) \label{option C}
\item \(\frac{8}{9}\) \label{option D}
\end{multicols}
\end{enumerate}
\item Solution\\
Given that X and Y are two random variables having joint PDF f(x,y)\\
The conditional PDF probability of two joint continuous random variable X and Y is given by-   
\begin{align}
f_{X|Y}\brak{x|y}=\frac{f_{XY}(x,y)}{f_Y(y)}
\end{align}
where f\textsubscript{Y}(y) is marginal PDF given by 
\begin{equation}
f_Y(y)=\int_{+\infty}^{-\infty} f_{XY}(x,y)dx \text{ ,  for all x} \in \mathbf{R}
\end{equation}
Here the marginal PDF of Y, lower limit of x is 0 and upper limit is y where  function is equal to 2( non zero value)
\begin{align}
f_Y(y)= \int_{0}^{y}2dy = 2y
\end{align}
and if x and y jointly continuous, then for any set A
\begin{align}
 \Pr(X \in A | Y = y) = \int_{A} f_{X|Y}(x|y)dx
\end{align}
so the conditional PDF is 
\begin{align}
f_{X|Y}\brak{x|y}= \begin{cases}
\frac{2}{2y}, \text{ if } 0<x<y<1\\
0, \text{ otherwise }
\end{cases}
\end{align}
the conditional probability X given Y=y is given by
\begin{equation}
\Pr(X\leqslant x|Y=y)=\int_{-\infty}^{x} f_{X|Y}(x|y)dx \text{ ,  for all x} \in \mathbf{R}  
\end{equation}
As Y=3/4 and lower limit of x is 0, so probablity is 
\begin{equation}
\Pr(X\leqslant x|Y=y)=\int_{0}^{2/3} \frac{4}{3}dx\\
\end{equation}
\begin{align}
\Pr(X\leqslant x|Y=y)=\frac{4}{3}\times \frac{2}{3}\\
\Pr(X\leqslant x|Y=y)=\frac{8}{9} 
\end{align}
So the correct option is \brak{d}
\end{enumerate}
\end{document}
