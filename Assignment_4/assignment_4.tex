  \documentclass[journal,12pt,twocolumn]{IEEEtran}

\usepackage{setspace}
\usepackage{gensymb}
\singlespacing
\usepackage[cmex10]{amsmath}

\usepackage{amsthm}

\usepackage{mathrsfs}
\usepackage{txfonts}
\usepackage{stfloats}
\usepackage{bm}
\usepackage{cite}
\usepackage{cases}
\usepackage{subfig}

\usepackage{longtable}
\usepackage{multirow}

\usepackage{enumitem}
\usepackage{mathtools}
\usepackage{steinmetz}
\usepackage{tikz}
\usepackage{circuitikz}
\usepackage{verbatim}
\usepackage{tfrupee}
\usepackage[breaklinks=true]{hyperref}
\usepackage{graphicx}
\usepackage{tkz-euclide}

\usetikzlibrary{calc,math}
\usepackage{listings}
    \usepackage{color}                                            %%
    \usepackage{array}                                            %%
    \usepackage{longtable}                                        %%
    \usepackage{calc}                                             %%
    \usepackage{multirow}                                         %%
    \usepackage{hhline}                                           %%
    \usepackage{ifthen}                                           %%
    \usepackage{lscape}     
\usepackage{multicol}
\usepackage{chngcntr}

\DeclareMathOperator*{\Res}{Res}

\renewcommand\thesection{\arabic{section}}
\renewcommand\thesubsection{\thesection.\arabic{subsection}}
\renewcommand\thesubsubsection{\thesubsection.\arabic{subsubsection}}

\renewcommand\thesectiondis{\arabic{section}}
\renewcommand\thesubsectiondis{\thesectiondis.\arabic{subsection}}
\renewcommand\thesubsubsectiondis{\thesubsectiondis.\arabic{subsubsection}}


\hyphenation{op-tical net-works semi-conduc-tor}
\def\inputGnumericTable{}                                 %%

\lstset{
%language=C,
frame=single, 
breaklines=true,
columns=fullflexible
}
\begin{document}


\newtheorem{theorem}{Theorem}[section]
\newtheorem{problem}{Problem}
\newtheorem{proposition}{Proposition}[section]
\newtheorem{lemma}{Lemma}[section]
\newtheorem{corollary}[theorem]{Corollary}
\newtheorem{example}{Example}[section]
\newtheorem{definition}[problem]{Definition}

\newcommand{\BEQA}{\begin{eqnarray}}
\newcommand{\EEQA}{\end{eqnarray}}
\newcommand{\define}{\stackrel{\triangle}{=}}
\bibliographystyle{IEEEtran}
\raggedbottom
\setlength{\parindent}{0pt}
\providecommand{\mbf}{\mathbf}
\providecommand{\pr}[1]{\ensuremath{\Pr\left(#1\right)}}
\providecommand{\qfunc}[1]{\ensuremath{Q\left(#1\right)}}
\providecommand{\sbrak}[1]{\ensuremath{{}\left[#1\right]}}
\providecommand{\lsbrak}[1]{\ensuremath{{}\left[#1\right.}}
\providecommand{\rsbrak}[1]{\ensuremath{{}\left.#1\right]}}
\providecommand{\brak}[1]{\ensuremath{\left(#1\right)}}
\providecommand{\lbrak}[1]{\ensuremath{\left(#1\right.}}
\providecommand{\rbrak}[1]{\ensuremath{\left.#1\right)}}
\providecommand{\cbrak}[1]{\ensuremath{\left\{#1\right\}}}
\providecommand{\lcbrak}[1]{\ensuremath{\left\{#1\right.}}
\providecommand{\rcbrak}[1]{\ensuremath{\left.#1\right\}}}
\theoremstyle{remark}
\newtheorem{rem}{Remark}
\newcommand{\sgn}{\mathop{\mathrm{sgn}}}
\providecommand{\abs}[1]{\(\left\vert#1\right\vert\)}
\providecommand{\res}[1]{\Res\displaylimits_{#1}} 
\providecommand{\norm}[1]{\(\left\lVert#1\right\rVert\)}
%\providecommand{\norm}[1]{\lVert#1\rVert}
\providecommand{\mtx}[1]{\mathbf{#1}}
\providecommand{\mean}[1]{E\(\left[ #1 \right]\)}
\providecommand{\fourier}{\overset{\mathcal{F}}{ \rightleftharpoons}}
%\providecommand{\hilbert}{\overset{\mathcal{H}}{ \rightleftharpoons}}
\providecommand{\system}{\overset{\mathcal{H}}{ \longleftrightarrow}}
	%\newcommand{\solution}[2]{\textbf{Solution:}{#1}}
\newcommand{\solution}{\noindent \textbf{Solution: }}
\newcommand{\cosec}{\,\text{cosec}\,}
\providecommand{\dec}[2]{\ensuremath{\overset{#1}{\underset{#2}{\gtrless}}}}
\newcommand{\myvec}[1]{\ensuremath{\begin{pmatrix}#1\end{pmatrix}}}
\newcommand{\mydet}[1]{\ensuremath{}}
\numberwithin{equation}{subsection}
\makeatletter
\@addtoreset{figure}{problem}
\makeatother
\let\StandardTheFigure\thefigure
\let\vec\mathbf
\renewcommand{\thefigure}{\theproblem}
\def\putbox#1#2#3{\makebox[0in][l]{\makebox[#1][l]{}\raisebox{\baselineskip}[0in][0in]{\raisebox{#2}[0in][0in]{#3}}}}
     \def\rightbox#1{\makebox[0in][r]{#1}}
     \def\centbox#1{\makebox[0in]{#1}}
     \def\topbox#1{\raisebox{-\baselineskip}[0in][0in]{#1}}
     \def\midbox#1{\raisebox{-0.5\baselineskip}[0in][0in]{#1}}
\vspace{3cm}
\title{AI1103 - Assignment 4}
\author{Vishwanath Hurakadli - AI20BTECH11023}
\maketitle
\newpage
\bigskip
\renewcommand{\thefigure}{\theenumi}
\renewcommand{\thetable}{\theenumi}
%
Download latex codes from 
%
\begin{lstlisting}
https://github.com/vishwahurakadli/AI1103/tree/main/Assignment_4
\end{lstlisting}
\begin{enumerate}
\item CSIR-UGC-NET(mathA june 2015) \\
Q.110: Suppose X has density
\begin{align}
f(x|\theta)= \frac{1}{\theta}e^{\frac{-x}{\theta}} , x>0,  \theta >0
\end{align}
Define Y as follows
\begin{align}
Y = K if k \leqslant X < k+1, k= 0,1,2,... 
\end{align}
Then the distribution of Y is
\begin{enumerate}
\item Normal 
\item Binomial 
\item Poisson 
\item Geometric
\end{enumerate}
\begin{lemma}
PDF of X is 
\begin{align}
\pr{X}&=\frac{1}{\theta} e^{-\frac{x}{\theta}}
\end{align}
\end{lemma}
\begin{proof}
The PDF of X will be given by
\begin{align}
f(x|\theta)&=\frac{1}{\theta} e^{-x/\theta},x>0,\theta >0\\
f(x)&=\frac{1}{\theta}e^{-\frac{x}{\theta}},  \text{since x and }\theta \text{are independent} \end{align}
Hence lemma 0.1 is proved.
\end{proof}
\begin{lemma}
pdf of Y is 
\begin{align}
p\brak{Y=k}&=e^{-\frac{k}{\theta}} \brak{1-e^{-\frac{k}{\theta}}}
\end{align}
is in the form of geometric distribution.
\end{lemma}
\begin{proof}
given that Y=k if $k \leqslant X <k+1$ k=0,1,2,...
\begin{align}
p(Y=k)&=\int_k^{k+1} p(X=x) dx\\
&=\int_k^{k+1} \frac{1}{\theta} e^{-\frac{x}{\theta}} dx \nonumber \\
&=e^{-\frac{k}{\theta}} \brak{1-e^{-\frac{k}{\theta}}}
\end{align}
\end{proof}
\begin{lemma}
When the pdf of X is in the form of 
\begin{align}
p(X=k)&=(1-p)^{k} p, \text{ where } (k=0,1,2,...)
\end{align} 
Then the distribution is said to be in the form of geometric distribution. 
\end{lemma}\\
Using Lemma 0.2 and Lemma 0.3, distribution of Y will be geometric.\\
So the correct option is \brak{4}
\end{enumerate}
\end{document}
