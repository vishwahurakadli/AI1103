  \documentclass[journal,12pt,twocolumn]{IEEEtran}

\usepackage{setspace}
\usepackage{gensymb}
\singlespacing
\usepackage[cmex10]{amsmath}

\usepackage{amsthm}

\usepackage{mathrsfs}
\usepackage{txfonts}
\usepackage{stfloats}
\usepackage{bm}
\usepackage{cite}
\usepackage{cases}
\usepackage{subfig}

\usepackage{longtable}
\usepackage{multirow}

\usepackage{enumitem}
\usepackage{mathtools}
\usepackage{steinmetz}
\usepackage{tikz}
\usepackage{circuitikz}
\usepackage{verbatim}
\usepackage{tfrupee}
\usepackage[breaklinks=true]{hyperref}
\usepackage{graphicx}
\usepackage{tkz-euclide}

\usetikzlibrary{calc,math}
\usepackage{listings}
    \usepackage{color}                                            %%
    \usepackage{array}                                            %%
    \usepackage{longtable}                                        %%
    \usepackage{calc}                                             %%
    \usepackage{multirow}                                         %%
    \usepackage{hhline}                                           %%
    \usepackage{ifthen}                                           %%
    \usepackage{lscape}     
\usepackage{multicol}
\usepackage{chngcntr}

\DeclareMathOperator*{\Res}{Res}

\renewcommand\thesection{\arabic{section}}
\renewcommand\thesubsection{\thesection.\arabic{subsection}}
\renewcommand\thesubsubsection{\thesubsection.\arabic{subsubsection}}

\renewcommand\thesectiondis{\arabic{section}}
\renewcommand\thesubsectiondis{\thesectiondis.\arabic{subsection}}
\renewcommand\thesubsubsectiondis{\thesubsectiondis.\arabic{subsubsection}}


\hyphenation{op-tical net-works semi-conduc-tor}
\def\inputGnumericTable{}                                 %%

\lstset{
%language=C,
frame=single, 
breaklines=true,
columns=fullflexible
}
\begin{document}


\newtheorem{theorem}{Theorem}[section]
\newtheorem{problem}{Problem}
\newtheorem{proposition}{Proposition}[section]
\newtheorem{lemma}{Lemma}[section]
\newtheorem{corollary}[theorem]{Corollary}
\newtheorem{example}{Example}[section]
\newtheorem{definition}[problem]{Definition}

\newcommand{\BEQA}{\begin{eqnarray}}
\newcommand{\EEQA}{\end{eqnarray}}
\newcommand{\define}{\stackrel{\triangle}{=}}
\bibliographystyle{IEEEtran}
\raggedbottom
\setlength{\parindent}{0pt}
\providecommand{\mbf}{\mathbf}
\providecommand{\pr}[1]{\ensuremath{\Pr\left(#1\right)}}
\providecommand{\qfunc}[1]{\ensuremath{Q\left(#1\right)}}
\providecommand{\sbrak}[1]{\ensuremath{{}\left[#1\right]}}
\providecommand{\lsbrak}[1]{\ensuremath{{}\left[#1\right.}}
\providecommand{\rsbrak}[1]{\ensuremath{{}\left.#1\right]}}
\providecommand{\brak}[1]{\ensuremath{\left(#1\right)}}
\providecommand{\lbrak}[1]{\ensuremath{\left(#1\right.}}
\providecommand{\rbrak}[1]{\ensuremath{\left.#1\right)}}
\providecommand{\cbrak}[1]{\ensuremath{\left\{#1\right\}}}
\providecommand{\lcbrak}[1]{\ensuremath{\left\{#1\right.}}
\providecommand{\rcbrak}[1]{\ensuremath{\left.#1\right\}}}
\theoremstyle{remark}
\newtheorem{rem}{Remark}
\newcommand{\sgn}{\mathop{\mathrm{sgn}}}
\providecommand{\abs}[1]{\(\left\vert#1\right\vert\)}
\providecommand{\res}[1]{\Res\displaylimits_{#1}} 
\providecommand{\norm}[1]{\(\left\lVert#1\right\rVert\)}
%\providecommand{\norm}[1]{\lVert#1\rVert}
\providecommand{\mtx}[1]{\mathbf{#1}}
\providecommand{\mean}[1]{E\(\left[ #1 \right]\)}
\providecommand{\fourier}{\overset{\mathcal{F}}{ \rightleftharpoons}}
%\providecommand{\hilbert}{\overset{\mathcal{H}}{ \rightleftharpoons}}
\providecommand{\system}{\overset{\mathcal{H}}{ \longleftrightarrow}}
	%\newcommand{\solution}[2]{\textbf{Solution:}{#1}}
\newcommand{\solution}{\noindent \textbf{Solution: }}
\newcommand{\cosec}{\,\text{cosec}\,}
\providecommand{\dec}[2]{\ensuremath{\overset{#1}{\underset{#2}{\gtrless}}}}
\newcommand{\myvec}[1]{\ensuremath{\begin{pmatrix}#1\end{pmatrix}}}
\newcommand{\mydet}[1]{\ensuremath{}}
\numberwithin{equation}{subsection}
\makeatletter
\@addtoreset{figure}{problem}
\makeatother
\let\StandardTheFigure\thefigure
\let\vec\mathbf
\renewcommand{\thefigure}{\theproblem}
\def\putbox#1#2#3{\makebox[0in][l]{\makebox[#1][l]{}\raisebox{\baselineskip}[0in][0in]{\raisebox{#2}[0in][0in]{#3}}}}
     \def\rightbox#1{\makebox[0in][r]{#1}}
     \def\centbox#1{\makebox[0in]{#1}}
     \def\topbox#1{\raisebox{-\baselineskip}[0in][0in]{#1}}
     \def\midbox#1{\raisebox{-0.5\baselineskip}[0in][0in]{#1}}
\vspace{3cm}
\title{AI1103 - Assignment 2}
\author{Vishwanath Hurakadli - AI20BTECH11023}
\maketitle
\newpage
\bigskip
\renewcommand{\thefigure}{\theenumi}
\renewcommand{\thetable}{\theenumi}
%
and latex codes from 
%
\begin{lstlisting}
https://github.com/vishwahurakadli/AI1103/blob/main/Assignment_2
\end{lstlisting}
\begin{enumerate}
\item Gate EC Q22:\\
Let U and V be two independent zero mean Gaussian random variables of variances \(\frac{1}{4}\) and \(\frac{1}{9}\) respectively. The probability \(\Pr\brak{3V\geqslant2U}\) is.
\begin{enumerate}[]
\begin{multicols}{4}
\setlength\itemsep{0.1em}

\item \(\frac{4}{9}\) \label{option A}
\item \(\frac{1}{2}\) \label{option B}
\item \(\frac{2}{3}\) \label{option C}
\item \(\frac{5}{9}\) \label{option D}
\end{multicols}
\end{enumerate}
\item Solution
Given that random variables U and V are zero mean Gaussian random variables
\begin{align}
i.e., E\brak{U}=E\brak{V}=0 
\end{align}
also given that
\begin{align}
 Var\brak{U}=\sigma^2_U=\frac{1}{4} \neq 0\\                                 Var\brak{V}=\sigma^2_V=\frac{1}{9} \neq 0
\end{align}
As U and V are Gaussian random variable
\begin{align}
f_u\brak{u}=\frac{1}{\sqrt{2\pi}\times\sigma^2_u}e^{{-u}/{2\sigma^2_u}}\\
f_v\brak{v}=\frac{1}{\sqrt{2\pi}\times\sigma^2_v}e^{{-v}/{2\sigma^2_v}}
\end{align}
As random variables are with zero mean and non zero variance by symmetry property imply that
\begin{align}
\Pr\brak{U\geqslant0}=\frac{1}{2}\\
\Pr\brak{V\geqslant0}=\frac{1}{2}
\end{align}
To interpret with 2U and 3V random variable, Let X and Y are random variables with following distribution 
\begin{align}
X = \frac{v}{\sigma_v}=3V\\
Y = \frac{u}{\sigma_u}=2U\\
\end{align}
and we have
\begin{align}
E\brak{X}=3E\brak{V}=0\\
E\brak{Y}=2E\brak{U}=0
\end{align}
imply that random variable X-Y
\begin{align}
E\brak{X-Y}=0
\end{align}
and variance of X and Y
\begin{align}
Var\brak{X}=9Var\brak{V}=1\\
Var\brak{Y}=4Var\brak{U}=1
\end{align}
imply that
\begin{align}
Var\brak{X-Y}\neq 0
\end{align}
Hence we can say by symmetry property
\begin{align}
\Pr\brak{X-Y\geqslant0}=\Pr\brak{3V-2U\geqslant0}=\frac{1}{2}  
\end{align}
So
\begin{align}
\Pr\brak{3V\geqslant2U}=\frac{1}{2}
\end{align}
So the correct option is \brak{B}
\end{enumerate}
\end{document}
